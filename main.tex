\documentclass{article}
\usepackage[utf8]{inputenc}
\usepackage[table]{xcolor}
\usepackage[rightcaption]{sidecap}
\usepackage{amsmath}
\usepackage{graphicx}
\usepackage[colorinlistoftodos]{todonotes}
\usepackage[colorlinks=true, allcolors=blue]{hyperref}
\usepackage[toc,page]{appendix}
\usepackage[utf8]{inputenc}

\bibliographystyle{stylename}
\bibliography{bibfile}
\usepackage{graphicx} %package to manage images
\usepackage{xcolor}
\newcommand\dubtes[1]{\textcolor{red}{#1}}
\graphicspath{ {images/} }
\title{Pràctica 0}
\usepackage{listings}
\usepackage{subfig}
\usepackage{wrapfig}

\newenvironment{ignora}{\color{grey}}{\ignorespacesafterend}

\begin{document}


\title{RESUM UX}

\maketitle{\textbf{Prefaci}}
\hfill
\textit{Això és un document orientatiu que resumeix The Elements of User Experience: User-Centered Design for the Web de Jesse James Garrett}
\hfill 

\section{Que és la UX i perquè importa.}
\hfill

Les males experiències són causades per mals dissenys, inclosa la cosa més petita forma part de la UX. La cosa no està en quin aspecte té o com funciona, si no en quin context l’usuari/ària utilitza això. 

Una bona UX és un ingredient essencial per desenvolupar lleialtat al producte (sigui el que sigui que vulgui dir això) i a grans trets una bona UX millora absolutament tot del producte.

\section{Els elements principals}
\hfill

Tenim 5 capes diferenciades: Superficial, esquelet, estructura, abast i estratègia.

Tot en una interacció en UX és intencional i és preferible construir respecte a les capes de baix (estratègia) cap amunt (superfície).

Un mal UX és causada per males decisions entre dues capes. S’ha de dir que la UX no està limitada per les pàgines web, qualsevol producte o servei inclou aquests elements.


\section{Estratègia}
\hfill

Una pàgina pot ser considerada un fracàs si no transmet el seu objectiu de manera correcta. Hem de respondre dues preguntes, que vol el negoci i que vol l’usuari i a partir d’aquestes dues preguntes hem d'arribar a una resposta que satisfaci a totes dues parts.

Les mètriques d’èxit són necessàries i per desgràcia s’apliquen poc, per exemple podem utilitzar Personas que són prototips d’usuaris que ens permeten tenir una simulació de com l’usuari/ària interacciona amb el producte.

\section{Abast}
\hfill

L’abast ens permet identificar el nostre camp de joc, possibles conflictes i tasques que hem de prioritzar, si no tenim un abast (objectiu) concret, seria com conduir a cegues sense un objectiu clar. Has de tenir en cap dues normes molt importants:Norma 1: Saber que vols construir
Norma 2: Saber que NO vols construir

S’ha de fer la distinció entre funcionalitat (les features del software) i el contingut (l’estructura en la qual representem la informació).

És necessari sempre fer una pluja d’idees, utilitzar persones, escenaris d’ús i copiar el que funciona dels teus rivals.

Important també CONTROLAR EL TEU ABAST, que no s’envagui de les mans, com més acotat estigui més senzill és el desenvolupament.

Les especificacions han de ser efectives i han de ser prou adaptables per treballar-hi. Claredat i precisió són molt més importants que el detall.  

Els elements més crítics han d'evitar confusió i no caure en una versió idealitzada del producte, per evitar això, segueix aquesta regla:Descriu que farà el producte, sigues específic en què serà exactament i evita llenguatge subjectiu


\section{Estructura}
\hfill

El software ha de ser més user friendly que no pas tècnicament eficient. La gestió d’errors juga un paper important en aquest aspecte i hem de procurar fer que sigui virtualment impossible que passi un error, hem de seguir el PCR: Prevenció, correcció i recuperació.

Tota mena de contingut ha d'estar estructurat en nodes independents que alhora es puguin relacionar entre ells, dintre d’això tenim diferents estructures per mostrar la informació: Jeràrquica, Matriu, Orgànica, Seqüencial

Per al que fa l’organització del contingut, hem de mantenir un cert vocabulari per mantenir consistència en el llenguatge i missatge, 


\section{Esquelet}
\hfill


Aquí ens centrem en el layout, la interacció i el flow que segueix l’usuari/ària en utilitzar l’aplicatiu. És l’art dels elements individuals dels components i la seva composició, entre els quals inclou:
- Disseny d’interfície
- Disseny de navegació
- Disseny de la informació

Cada un d’aquests depenen dels altres. Les convencions existeixen per un motiu, ja que ens ajuden a mantenir una consistència en el producte. Hem d'estructurar el layout i la interfície per donar suport a les tasques que l’usuari/ària ha de fer i ressaltar el que és important.

\section{Superfície}
\hfill



Aquí ens ocupem del arrengement de la informació i com ha de ser presentada en l'àmbit visual.

En comptes de centrar-nos en el que és agradable a la vista ens hem de centrar en com és d'efectiu el disseny per donar suport als objectius definits per cadascuna de les capes anteriors.

Una manera efectiva d'analitzar això és seguir la vista, en obrir un aplicatiu on va a parar l'ull inicialment?

Una altra eina que podem usar és el contrast que ens permet veure les relacions entre els diferents elements de navegació de la pàgina.

Hem de mantenir uniformitat per evitar confusions o saturació de la informació. Un exemple és tenir uns.


\section{Aplicació dels elements}
\hfill





Per aplicar tota la informació anterior de manera integrada un approach senzill que ens pot ajudar és respectar aquestes dues idees:

1. Entendre quin problema estem intentant solucionar. En quina capa realment està el problema? Si volem moure un botó és la part visual que no ens agrada (superfície) o la funció del botó no és la que esperen els usuaris (estructura)

2. Entendre les conseqüències de la solució proposada. Pot trencar la lògica interna que tenim amb el producte.

Una decisió petita pot trencar el procés de disseny.

Per desgràcia la UX és la primera víctima en un entorn de sprint i desenvolupament ràpid, per això hem d'aplicar aquests principis en cada pas que fem i assegurar-nos que el producte compleix tant amb els objectius d’estratègia com les necessitats de l'usuari/ària.



\begin{thebibliography}{9}

\bibitem{Webinar}
The Elements of User Experience: User-Centered Design for the Web de Jesse James Garrett

https://www.librouro.com/libro/ver/1354582-the-elements-of-user-experience-usercentered-design-for-the-web.html

\end{thebibliography}

\end{document}

